\section{\large Технологическая часть}

\subsection*{Введение}

В данном разделе представлены требования к программе и реализованы спроектированные методы.

\subsection{Требования к программе}

Для того чтобы программное обеспечение удовлетворяло требованиям, необходимо определить их заранее и придерживаться их в процессе разработки. 
Программное обеспечение должно удовлетворять требованиям, которые необходимы для работы спроектированной системы:

\begin{enumerate}
    \item поддержка запросов GET и HEAD;
    \item корректная передача файлов размером в 100мб;
    \item поддержка многопоточности.
\end{enumerate}

\subsection{Средства реализации}

Для реализации ПО был выбран язык C \cite{C}.
В данном языке есть все требующиеся инструменты для данной курсовой работы.
В качестве среды разработки была выбрана среда Clion \cite{vscode}.

\subsection{Реализация сокета poll}

\begin{lstlisting}[label=dbcreate, caption=Сокета poll]
    while (1)
	{
		numfds = poll(server->clients, maxcl + 1, -1);
		if (numfds < 0)
		{
			LOG_ERROR("poll error");
			continue;
		}
		if (server->clients[0].revents & POLLIN)
		{
			int client_sock = accept(server->listen_sock, NULL, 0);
			if (client_sock < 0) continue;
			long i = 0;
			for (i = 1; i < server->cl_num; ++i)
			{
				if (server->clients[i].fd < 0)
				{
					server->clients[i].fd = client_sock;
					server->clients[i].events = POLLIN | POLLPRI;
					break;
				}
			}
			if (i == server->cl_num) {
				LOG_ERROR("too many connections");
				continue;
			}
			if (i > maxcl)
			{
				maxcl = i;
				LOG_INFO("Max clients: %d", maxcl);
			}
			if (--numfds <= 0)
			{
				continue;
			}
		}
		for (int i = 1; i <= maxcl; ++i)
		{
			if (server->clients[i].fd >= 0 && server->clients[i].revents & (POLLIN | POLLERR))
			{
				worker_sock_t worker_sock;
				worker_sock.clientfd = &server->clients[i].fd;
				worker_sock.wd = server->wd;
				tpool_add_work(server->pool, worker, &worker_sock);

				if (--numfds < 0)
				{
					break;
				}
			}
		}
		tpool_wait(server->pool);
	}
\end{lstlisting}

\subsection{Реализация пуллинг потоков}

\begin{lstlisting}[label=dbcreate, caption=Сокета poll]
    static void* tpool_worker(void* arg)
    {
        tpool_t* tm = arg;
        tpool_work_t* work;
    
        while (1)
        {
            pthread_mutex_lock(&(tm->work_mutex));
            while (tm->work_first == NULL && !tm->stop)
                pthread_cond_wait(&(tm->work_cond), &(tm->work_mutex));
            if (tm->stop)
                break;
            work = tpool_work_get(tm);
            tm->working_cnt++;
            pthread_mutex_unlock(&(tm->work_mutex));
            if (work != NULL)
            {
                work->func(work->arg);
                tpool_work_destroy(work);
            }
            pthread_mutex_lock(&(tm->work_mutex));
            tm->working_cnt--;
            if (!tm->stop && tm->working_cnt == 0 && tm->work_first == NULL)
                pthread_cond_signal(&(tm->working_cond));
            pthread_mutex_unlock(&(tm->work_mutex));
        }
        tm->thread_cnt--;
        pthread_cond_signal(&(tm->working_cond));
        pthread_mutex_unlock(&(tm->work_mutex));
        return NULL;
    }    
\end{lstlisting}


\subsection*{Вывод}

В данном разделе представлены требования к программе и реализованы спроектированные методы.